\documentclass[12pt]{article}
\title{Project Interviews}
\author{Bertagnoli Daniele 1903768 \\ Costantino Giuseppe \\ Frascarelli Giacomo \\ Guarino Marco}
\date{2023/2024}

\usepackage[left=2cm, right=2cm]{geometry}
\usepackage{amsmath}
\usepackage{graphicx}
\usepackage{subfig}

\begin{document}
\maketitle

\tableofcontents
\newpage

\section{Interviews}
These interviews were conducted to assess the viability of the project idea and serve as the initial stage of the need-finding process. The questions asked to the participants during the interview are:
\begin{enumerate}
    \item Have you experienced situations in which you or someone you know was in danger? (e.g., my father fell down and was not able to use the phone) If yes, what was the situation? Were you able to contact someone for help? Was it difficult, or were you able to alert someone easily?
    \item Would you install the application to alert contacts in an easy and fast way? Do you find this idea useful?
    \item Do you prefer to inform contacts through a phone call, SMS, or a notification? (Or other options)
    \item Would you install such an application to be informed about other contacts' emergencies? Why?
\end{enumerate}

The following rows contain the main content of each interview, highlighting the most significant contributions given by each participant.

\begin{enumerate}
    \item She said that once her grandmother had a medical emergency at 
    home, and she wasn't able to reach anyone for help because her phone 
    was out of reach. However, the grandmother is not able to use the 
    smartphone.
    
    \item He shared a story about a camping trip where one of his friends got 
    injured, and his phone fell several meters away from him. He started screaming 
    to ask for help, luckily, the other friends heard him. He said that he would 
    install that application only during particular periods of the year, such as when 
    he goes on holidays during summer. He prefers phone calls whenever possible.
    
    \item He mentioned a situation where a colleague experienced a severe allergic 
    reaction while he was alone at home. He was able to use the phone, but he 
    struggled to talk during the phone call with the ambulance operators. He said 
    that probably a method to alert family members in some ways with predefined 
    messages was probably the best thing in that situation because he had to alert 
    them several hours later when the crisis was gone. He told us that these 
    situations are not so common, so probably he wouldn't install such an application 
    as he prefers to be contacted via WhatsApp or SMS through that system.
    
    \item She told us about one situation in which she felt followed by someone. 
    Because she was very nervous, she had some difficulties quickly dialing her 
    mother's phone number on the phone. She stated that she would install the 
    application to feel safer when she comes back home during late hours of the day.
    
    \item He doesn't know anyone who experienced emergencies. He said that this is a 
    situation he has never thought about, but the application could be useful for his 
    grandma who lives alone and uses a smartphone. The interviewee prefers to be 
    informed via notifications.
\end{enumerate}

\end{document}