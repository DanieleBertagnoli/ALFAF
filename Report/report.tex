\documentclass[12pt]{article}
\title{Project Interviews}
\author{Bertagnoli Daniele 1903768 \\ Costantino Giuseppe \\ Frascarelli Giacomo \\ Guarino Marco}
\date{2023/2024}

\usepackage[left=2cm, right=2cm]{geometry}
\usepackage{amsmath}
\usepackage{graphicx}
\usepackage{subfig}

\begin{document}
\maketitle

\tableofcontents
\newpage

\section{Interviews}
These interviews were conducted to assess the viability of the project idea and serve as the initial stage of the need-finding process. The questions asked to the participants during the interview are:
\begin{enumerate}
    \item Have you experienced situations in which you or someone you know was in danger? (e.g., my father fell down and was not able to use the phone) If yes, what was the situation? Were you able to contact someone for help? Was it difficult, or were you able to alert someone easily?
    \item Would you install the application to alert contacts in an easy and fast way? Do you find this idea useful?
    \item Do you prefer to inform contacts through a phone call, SMS, or a notification? (Or other options)
    \item Would you install such an application to be informed about other contacts' emergencies? Why?
\end{enumerate}

The following rows contain the main content of each interview, highlighting the most significant contributions given by each participant.

\begin{enumerate}
    \item She said that once her grandmother had a medical emergency at 
    home, and she wasn't able to reach anyone for help because her phone 
    was out of reach. However, the grandmother is not able to use the 
    smartphone.
    
    \item He shared a story about a camping trip where one of his friends got 
    injured, and his phone fell several meters away from him. He started screaming 
    to ask for help, luckily, the other friends heard him. He said that he would 
    install that application only during particular periods of the year, such as when 
    he goes on holidays during summer. He prefers phone calls whenever possible.
    
    \item He mentioned a situation where a colleague experienced a severe allergic 
    reaction while he was alone at home. He was able to use the phone, but he 
    struggled to talk during the phone call with the ambulance operators. He said 
    that probably a method to alert family members in some ways with predefined 
    messages was probably the best thing in that situation because he had to alert 
    them several hours later when the crisis was gone. He told us that these 
    situations are not so common, so probably he wouldn't install such an application 
    as he prefers to be contacted via WhatsApp or SMS through that system.
    
    \item She told us about one situation in which she felt followed by someone. 
    Because she was very nervous, she had some difficulties quickly dialing her 
    mother's phone number on the phone. She stated that she would install the 
    application to feel safer when she comes back home during late hours of the day.
    
    \item He doesn't know anyone who experienced emergencies. He said that this is a 
    situation he has never thought about, but the application could be useful for his 
    grandma who lives alone and uses a smartphone. The interviewee prefers to be 
    informed via notifications.

    \item She said her mother fell to the ground in the evening, couldn't 
    get up, and couldn't call anyone. She would install an application that 
    could help herself and others in difficult situations. She 
    would like to be notified both by a call and by an SMS.
    
    \item She says she was alone in the middle of the night in Rome and felt 
    like someone was chasing her. So, she was on the phone the whole time 
    with her friend. She would like an application that can promptly assist 
    her in case of difficulty with a call. Additionally, she would install the 
    application to help other people in emergency situations.
    
    \item He has never experienced firsthand situations where he was in trouble 
    or where someone he knows alerted him in emergencies, but he would still 
    find an application for such purposes useful and would install it. 
    He would like to be notified, and to notify others, using all available 
    methods.
    
    \item He experienced a situation where his father had a heart attack 
    and fortunately managed to contact the ambulance but couldn't reach 
    his son, who only found out about the news after some time. So, he 
    would appreciate having the possibility of being contacted immediately. 
    He would thus install the application both in case he himself is in 
    danger and in case it's one of his parents. He would like to be 
    informed with a call and a notification of the location.
    
    \item She has never experienced situations of danger herself nor 
    secondhand where it was necessary to contact someone, but she would 
    appreciate an application that can alert others in a simplified way, 
    and to be alerted, in cases of danger and emergency. She would find it 
    useful to be notified in all available ways, through calls, SMS, and 
    notifications. She would install the application both for herself and 
    for friends or relatives.

    \item The interviewee had a nasty situation once while riding his bike. 
    He slipped while making a turn, and his smartphone flew out of his pocket. 
    Both his legs were stuck underneath the bike, and he was unable to move. 
    It wasn't until a good samaritan came by that he received medical 
    attention. He mentioned that had it been an isolated road with bad 
    weather, he could have easily died that day. He thinks calls are the 
    most effective way to get someone's attention, though a very loud 
    notification from the app could also work. He expressed interest in 
    using the app because his parents are getting older, and he would love 
    to keep an eye on them to be notified immediately in case of an 
    emergency.

    \item This person recounted an incident where she was trying to get her 
    grandfather to bed when he suddenly lost consciousness and fell on his 
    back. She immediately called the emergency number and got him escorted 
    to the hospital. She noted that her grandfather does not even have a 
    clamshell phone and is not keen on integrating technology into his 
    daily life. Thus, she does not think this idea could benefit her 
    family.

    \item She has had a few close calls while rock climbing. One time, she 
    fell and got stuck in a narrow crevice. A fellow climber nearby was able 
    to pull her out safely. This experience made her realize the importance 
    of a reliable emergency response system. She believes this idea could 
    benefit people with disabilities or mobility issues by providing an easy 
    way to alert contacts in case of an emergency. She thinks SMS messages 
    would be the most effective notification method to ensure the person 
    receiving the message is aware of the situation. She would definitely 
    install an app like this to keep an eye on loved ones who live far away, 
    understanding the importance of seniors having autonomy while still 
    having a safety net.

    \item He shared that his mum once fell down the stairs, which was quite 
    worrying for him and his brother. They managed to patch her up, and she 
    walked away with a minor scratch. While he finds the idea useful, he 
    does not trust technology enough. He worries about the phone's battery 
    running out, lack of signal, or a bug preventing a notification from 
    appearing. He prefers good old-fashioned phone calls and does not 
    believe apps can truly benefit daily life.

    \item The interviewee has never had a personal experience with danger 
    but has worked in emergency services for years. He has seen the impact 
    of timely interventions on people's lives, with one incident involving a 
    distress call from a lost hiker. He managed to locate and guide the 
    hiker back to safety, which highlighted the importance of having the 
    right tools and systems in place. He believes efficient communication is 
    crucial and views the app as a game-changer for staying connected and 
    getting help quickly. He thinks a combination of notifications and phone 
    calls would be most effective for alerting contacts in an emergency. 
    He would install the app to receive notifications about other contacts' 
    emergencies, believing in the importance of being connected and informed 
    to help each other, whether by providing support or offering a listening 
    ear.

\end{enumerate}

\end{document}