\documentclass[12pt]{article}
\title{Project Interviews}
\author{Bertagnoli Daniele 1903768 \\ Costantino Giuseppe \\ Frascarelli Giacomo \\ Guarino Marco}
\date{2023/2024}

\usepackage[left=2cm, right=2cm]{geometry}
\usepackage{amsmath}
\usepackage{graphicx}
\usepackage{subfig}

\begin{document}
\maketitle

\tableofcontents
\newpage

\section{Interviews}
These interviews were taken to understand if the project was a valid idea. We used them also as the first part of the need-finding step. The questions asked to the participants during the interview are:
\begin{enumerate}
    \item Have you got any old-aged family member? If yes, does he/she use the smartphone?
    \item Have you experienced some situations in which that person was in danger? (e.g. he/she fell down and was not able to use the phone) If yes, how did you solve the problem?
    \item Do you think that he/she will be fine in using an application to be used in such emergencies? Why yes? Why not?
    \item Would you install the application to communicate with him/her or do you prefer to be informed via SMS or phone call? Why?
    \item Do you think that such an application could be useful even for younger people or not? Why?
\end{enumerate}

The following rows are the main content of each interview, to highlight the most important contribution given by each of those.

\begin{enumerate}
    \item The interviewee said his grandma, 65, struggles with her smartphone 
    but manages basic functions. Once, she fell and couldn't reach her phone. A 
    neighbor luckily helped. He's unsure about an emergency app for her, 
    fearing it might confuse her more. He doubts its usefulness for 
    younger folks too, unless special medical conditions.
    
    \item The participant shared her dad, 70s, is good with his smartphone. 
    Once, he felt chest pain but couldn't dial due to panic. Luckily, she 
    was at home at that moment. She thinks an app could help but worries about 
    its complexity. She prefers calls/SMS for simplicity. She thinks that the application 
    could be even more useful for younger peaple as they are more practical with the 
    smartphone.
    
    \item Her grandpa, 80, does not use the smartphone. However, she's trying to 
    teach him to use the classical phone at least for the basic functions. 
    She said that probably if the grandpa was able to use the smartphone, such 
    application could be very useful. She doesn't have a preferred notification 
    method.
    
    \item He mentioned that he hasn't any old-aged family member. However, his 
    brother, who is 35, has a special medical condition (he preferred to keep it 
    private) that could faint him in any moment. If the application was able 
    to detect an eventual fall and also specify the brother's position, it 
    would be perfect for their situation. He would like to be notified using SMS 
    or through the application as, trivially, when the brother faints he's not 
    able to speak. 
    
    \item His grandma, 80, doesn't use a smartphone due to tech difficulties. 
    Once, she had a medical emergency but couldn't call. A family member arrived in time. 
    He doubts she would use an app (even in the hypothesis of being able to use the 
    smartphone), thinking it would confuse her. He prefers calls/SMS. 
    He's uncertain about its usefulness for younger people too.
\end{enumerate}

\end{document}